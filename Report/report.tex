\documentclass{article}
\usepackage{xurl}
\usepackage{hyperref}
\hypersetup{colorlinks=true, linkcolor=blue, urlcolor=blue}
\usepackage{geometry}
\setlength{\parindent}{0pt}
\title{\Huge Khidmat Report\\
\textit{WWF Bird Recognition Model}}
\author{Student 1: Ali Hamza\\ ah05084
  \and
  Student 2: Ali Haider\\ xy09876
  \and
  Student 3: Usaid Rehman\\ xy09876
}
\date{\today}  
\begin{document}
\maketitle
\newpage
\tableofcontents
\newpage
\section{Introduction}
% Use first person plural (we, us) even if you did the Khidmat individually.
\subsection{Project Description}
\setlength{\parindent}{3em}
% An introduction of the project, no more than 2 sentences. Provide the highest level of detail only. Other details will come later.
% Typically, "This project is to <short description of porject> for/at <client>."
This project is aimed at creating and deploying a deep learning pipeline for WWF Pakistan to classify images of 
three different species of birds -- namely: common myna, house crow, and the house sparrow. This project 
would serve as a proof-of-concept for a larger model that WWF can use to classify a larger number of birds 
using a mobile application.

\subsection{About WWF}
% About the client.
\subsection{Work Plan}
% About the plan of work.
\newpage % Start the report for each week on a new page.

% Copy-paste this section with necessary modifcations for each week.
\section{Weekly Work Log}
\subsection{\texttt{Week 1: }}
%Short Weekly Summary
\begin{center}
\begin{tabular}{|l|l|l|l|}
  \hline
  \textbf{Item} 	& \textbf{Activity} & \textbf{Time} & \textbf{ID} \\\hline
  
\end{tabular}
    
\end{center}
The total time spent on the Khidmat this week is as follows.    
\begin{center}
    
\begin{tabular}{|l|l|}
  \hline
  \textbf{ID} & \textbf{Total Hours}\\\hline
  st1 & \\\hline
  st2 & \\\hline
  st3 & \\\hline
\end{tabular}
\end{center}
\newpage
\subsection{\texttt{Week 2: }}
%Short Weekly Summary
\begin{center}
\begin{tabular}{|l|l|l|l|}
  \hline
  \textbf{Item} 	& \textbf{Activity} & \textbf{Time} & \textbf{ID} \\\hline
  
\end{tabular}
    
\end{center}
The total time spent on the Khidmat this week is as follows.    
\begin{center}
    
\begin{tabular}{|l|l|}
  \hline
  \textbf{ID} & \textbf{Total Hours}\\\hline
  st1 & \\\hline
  st2 & \\\hline
  st3 & \\\hline
\end{tabular}
\end{center}
\newpage
\subsection{\texttt{Week 3: }}
%Short Weekly Summary
\begin{center}
\begin{tabular}{|l|l|l|l|}
  \hline
  \textbf{Item} 	& \textbf{Activity} & \textbf{Time} & \textbf{ID} \\\hline
  
\end{tabular}
    
\end{center}
The total time spent on the Khidmat this week is as follows.    
\begin{center}
    
\begin{tabular}{|l|l|}
  \hline
  \textbf{ID} & \textbf{Total Hours}\\\hline
  st1 & \\\hline
  st2 & \\\hline
  st3 & \\\hline
\end{tabular}
\end{center}
\newpage
\subsection{\texttt{Week 4: }}
%Short Weekly Summary
\begin{center}
\begin{tabular}{|l|l|l|l|}
  \hline
  \textbf{Item} 	& \textbf{Activity} & \textbf{Time} & \textbf{ID} \\\hline
  
\end{tabular}
    
\end{center}
The total time spent on the Khidmat this week is as follows.    
\begin{center}
    
\begin{tabular}{|l|l|}
  \hline
  \textbf{ID} & \textbf{Total Hours}\\\hline
  st1 & \\\hline
  st2 & \\\hline
  st3 & \\\hline
\end{tabular}
\end{center}
\newpage
\subsection{\texttt{Week 5: }}
%Short Weekly Summar
\begin{center}
\begin{tabular}{|l|l|l|l|}
  \hline
  \textbf{Item} 	& \textbf{Activity} & \textbf{Time} & \textbf{ID} \\\hline
  
\end{tabular}
    
\end{center}
The total time spent on the Khidmat this week is as follows.    
\begin{center}
    
\begin{tabular}{|l|l|}
  \hline
  \textbf{ID} & \textbf{Total Hours}\\\hline
  st1 & \\\hline
  st2 & \\\hline
  st3 & \\\hline
\end{tabular}
\end{center}
\newpage
\subsection{\texttt{Week 6: }}
%Short Weekly Summary
\begin{center}
\begin{tabular}{|l|l|l|l|}
  \hline
  \textbf{Item} 	& \textbf{Activity} & \textbf{Time} & \textbf{ID} \\\hline
  
\end{tabular}
    
\end{center}
The total time spent on the Khidmat this week is as follows.    
\begin{center}
    
\begin{tabular}{|l|l|}
  \hline
  \textbf{ID} & \textbf{Total Hours}\\\hline
  st1 & \\\hline
  st2 & \\\hline
  st3 & \\\hline
\end{tabular}
\end{center}
\newpage
\subsection{\texttt{Week 7: }}
%Short Weekly Summary
\begin{center}
\begin{tabular}{|l|l|l|l|}
  \hline
  \textbf{Item} 	& \textbf{Activity} & \textbf{Time} & \textbf{ID} \\\hline
  
\end{tabular}
    
\end{center}
The total time spent on the Khidmat this week is as follows.    
\begin{center}
    
\begin{tabular}{|l|l|}
  \hline
  \textbf{ID} & \textbf{Total Hours}\\\hline
  st1 & \\\hline
  st2 & \\\hline
  st3 & \\\hline
\end{tabular}
\end{center}
\newpage
\subsection{\texttt{Week 8: }}
%Short Weekly Summary
\begin{center}
\begin{tabular}{|l|l|l|l|}
  \hline
  \textbf{Item} 	& \textbf{Activity} & \textbf{Time} & \textbf{ID} \\\hline
  
\end{tabular}
    
\end{center}
The total time spent on the Khidmat this week is as follows.    
\begin{center}
    
\begin{tabular}{|l|l|}
  \hline
  \textbf{ID} & \textbf{Total Hours}\\\hline
  st1 & \\\hline
  st2 & \\\hline
  st3 & \\\hline
\end{tabular}
\end{center}
\newpage
\section{Conclusion}
\newpage

\section{Technical Overview}
\subsection{Resources}
  The resources that we used for this project can be categorized into two main 
  categories: images, and technologies. Our first step was to acquire a set of images that would 
  function as our dataset. We collected around 550 images for each of the three 
  species of birds --- the house crow, common myna, and the house sparrow. We used 
  several different websites and online databases to find these images, we list 
  these websites below: 
  \begin{enumerate}
    \item \url{https://search.macaulaylibrary.org/catalog?taxonCode=myna&mediaType=p&q=Common\%20Myna}
    \item \url{https://ebird.org/media/catalog?taxonCode=commyn&mediaType=p&sort=rating_rank_desc&q=Common\%20Myna\%20-\%20Acridotheres\%20tristis}
    \item \url{https://ebird.org/media/catalog?taxonCode=houcro1&sort=rating_rank_desc&mediaType=p&regionCode=}
        \item \url{https://search.macaulaylibrary.org/catalog?taxonCode=houcro1&mediaType=p&region=Pakistan\%20(PK)&regionCode=PK&q=House\%20Crow\%20-\%20Corvus\%20splendens}
        \item \url{https://www.kaggle.com/gpiosenka/100-bird-species}
        \item \url{https://search.macaulaylibrary.org/catalog?taxonCode=houspa&mediaType=p&q=House\%20Sparrow}
          \item \url{https://ebird.org/media/catalog?taxonCode=houspa&mediaType=p&sort=rating_rank_desc&q=House\%20Sparrow\%20-\%20Passer\%20domesticus}
  \end{enumerate}
  \setlength{\parskip}{1em}
  All collected images can be viewed at \url{https://drive.google.com/drive/folders/18k-roE_VJSB1dcrhvN1y_EosVF7Kb0dY?usp=sharing}.

  To preprocess our images, convert them into a usable dataset, and to create
  our deep learning model, we had to rely on several different preexisting tools.
  We list the tools that we used in this project below:
  \setlength{\parskip}{0em}
  \begin{itemize}
    \item \texttt{Python} --- The primary programming language that we used for this project.
      Most of the other tools that we used are different libraries \& modules for in Python.
    \item \texttt{OpenCV} --- A computer vision module in Python to preprocess the images. 
    \item \texttt{TensorFlow \& Keras} --- A Python library that we used to construct, and then 
      ttrain and test our neural network. 
    \item \texttt{Matplotlib} --- A plotting library for Python, which we used to visualize our
    \item \texttt{Sci-kit learn} --- A Python library for machine learning that we used to 
      construct our training and testing set. 
    \item \texttt{Google Colab} --- We used Colab notebooks to write code and then run code
      using GPU-accelerated computation that is made available using Google Colaboratory.
  \end{itemize}
 
\subsection{Overview of CS Techniques Used}
  This project was an amalgamation of several different aspects of computer science techniques.
  The process can be summarized into 5 main steps:
  \begin{enumerate}
    \item Image Collection
    \item Image Preprocessing 
    \item Construction of Neural Network 
    \item Training the Neural Network
    \item Testing and Optimization
  \end{enumerate}
  
  We briefly expand upon these steps in the following sections. For a more technical 
  description, please view the documentation (LINK HERE). 

  \subsubsection{Image Collection}
   
  \subsubsection{Image Preprocessing}
  
  \subsubsection{Neural Network Construction}
  
  \subsubsection{Training the Neural Network}

  \subsubsection{Testing and Optimization}

\newpage
\appendix
\section{Source Code}
\newpage
\thispagestyle{empty}
% Show your external supervisor your report, especially the weekly upates; have them sign a printed copy of this page; scan the signed page; and include the scanned page in this document as an image.
\section{Documentation}
\newpage
\begin{center}
  {\Large\bf Khidmat Completion Form}\\[5pt]
  \small To be completed by the external supervisor.  
\end{center}
\bigskip

\noindent{\it Please use the space below to provide any comments you may have on the students' performance, the Khidmat program, or any other feedback you want to share with Habib University's Khidmat committee. We can also be reached at \href{mailto:khidmat@sse.habib.edu.pk}{khidmat@sse.habib.edu.pk}.}
\vfill

\begin{center}
  \rule{.8\textwidth}{.5pt}
\end{center}
\medskip

% Insert your name below.

I hereby certify that I supervised XXX and XXX for the Khidmat described in this report. Furthermore, that I have read and agree with the weekly updates included in this report. My signature below marks the successful completion of the work undertaken for the Khidmat.\\
\bigskip
\bigskip

\noindent\begin{tabular}{@{}p{.6\textwidth}@{\hspace{.1\textwidth}}p{.3\textwidth}}
  \hrulefill \&   \hrulefill\\
  Name and signature & Location and date
\end{tabular}

\end{document}
